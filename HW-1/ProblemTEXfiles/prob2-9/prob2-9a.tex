\documentclass[ClusteringConnectionsMAIN.tex]{subfiles}




\begin{document}
	

\prob{2.9 a}  Let $\rho \left( X, Y \right)$ be a function which is defined accoring to

\begin{align} \label{c2p9eq1}
\rho \left( X, Y \right) = H \left( X \mid Y \right)+ H \left( Y \mid X \right)  \tag{2.9-1}
\end{align}

for all $x$ and $y$.  Since conditional probabilites are always non-zero (\emph{for arbitrary} $X$ \emph{and} $Y$ \emph{we have} $H \left( X \mid Y \right) \geq 0$), we can say that $H \left( X \mid Y \right)$ has the property

\begin{align*}
H \left( X \mid Y \right) \geq 0
\end{align*} 

and that $H \left( Y \mid X \right)$ has the property

\begin{align*}
H \left( Y \mid X \right) \geq 0
\end{align*} 

Applying these properties of $H \left( X \mid Y \right) \geq 0$ and $H \left( Y \mid X \right) \geq 0$ to the expression in \ref{c2p9eq1} yields

\begin{align} \label{c2p9eq2}
\rho \left( X, Y \right) = H \left( X \mid Y \right)+ H \left( Y \mid X \right)  \, \geq 0  \tag{2.9-2}
\end{align}

which indicates that $\rho \left( X, Y \right)$ satisfies the first property of a metric over all $x$ and $y$.  By its definition in \ref{c2p9eq1}, we can say that $\rho \left( X, Y \right)$ is symmetric; therefore, we can additionally say that $\rho \left( X, Y \right)$ satisfies the second condition of a metric over all $x$ and $y$. \newline


Now, consider three random variables, $X, Y$, and $Z$.  Then write

\begin{align} \label{c2p9eq3}
\rho \left( X, Y \right) = H \left( X \mid Y \right)+ H \left( Y \mid X \right)  \tag{2.9-3}
\end{align}

and

\begin{align} \label{c2p9eq4}
\rho \left( Y, Z \right) = H \left( Y \mid Z \right)+ H \left( Y \mid Z \right)  \tag{2.9-4}
\end{align}

and

\begin{align} \label{c2p9eq5}
\rho \left( X, Z \right) = H \left( X \mid Z \right)+ H \left( Z \mid X \right)  \tag{2.9-5}
\end{align}

We now add the expression in \ref{c2p9eq3} and \ref{c2p9eq4} to obtain

\begin{align} \label{c2p9eq6}
H \left( X \mid Y \right)+ H \left( Y \mid X \right) + H \left( Y \mid Z \right)+ H \left( Z \mid Y \right)  \notag \\
\biggl[ H \left( X \mid Y \right) + H \left( Y \mid Z \right) \biggr] + \biggl[ H \left( Z \mid Y \right) + H \left( Y \mid X \right) \biggr]  \tag{2.9-6}
\end{align}

By the chain rule for conditional entropies, we have $H \left( X \mid Y \right) + H \left( Y \mid Z \right) = H \left( X, Y \mid Z \right)$ and $H \left( Z \mid Y \right) + H \left( Y \mid X \right) = H \left( Z, Y \mid X \right)$ so the expression in \ref{c2p9eq6} becomes

\begin{align} 
H \left( X, Y \mid Z \right) + H \left( Z, Y \mid X \right)  \notag
\end{align}

Again appling the chain rule for conditional entropies to the previous result, we have

\begin{align} 
\biggl[ H \left( X \mid Z \right) + H \left( Y \mid X, Z \right) \biggr] + \biggl[ H \left( Z \mid X \right) + H \left( Y \mid Z, X \right) \biggr]   \notag
\end{align}

SInce conditional entropies are always greater or equal to zero, the $H \left( Y \mid X, Z \right)$ and $H \left( Y \mid Z, X \right)$ terms in the previous result satisfy $H \left( Y \mid X, Z \right) \geq 0$ and $H \left( Y \mid Z, X \right) \geq 0$.  This allows us to rewrite the previous result as

\begin{align} 
\rho \left( X, Y \right) + \rho \left( Y, Z \right) &= H \left( Y \mid Z \right)+ H \left( Y \mid Z \right) +  H \left( X \mid Y \right)+ H \left( Y \mid X \right)  \notag \\
&= \biggl[ H \left( X \mid Z \right) + H \left( Y \mid X, Z \right) \biggr] + \biggl[ H \left( Z \mid X \right) + H \left( Y \mid Z, X \right) \biggr] \notag \\
&\geq H \left( X \mid Z \right) + H \left( Z\mid X \right) \notag 
\end{align}

Using the definition from \ref{c2p9eq5}, the previous result becomes

\begin{align} 
\rho \left( X, Y \right) + \rho \left( Y, Z \right) &= \biggl[ H \left( X \mid Z \right) + H \left( Y \mid X, Z \right) \biggr] + \biggl[ H \left( Z \mid X \right) + H \left( Y \mid Z, X \right) \biggr] \notag \\
&\geq H \left( X \mid Z \right) + H \left( Z\mid X \right) \notag \\
&\geq \rho \left( X, Z \right)  \notag
\end{align}

thereby indicating that $\rho \left( X, Y \right) + \rho \left( Y, Z \right) \geq \rho \left( X, Z \right)$ holds and, by extension, that the definition in \ref{c2p9eq1} satisfies the fourth condition of a metric over all $x$ and $y$. \newline


Finally, we consider the case where $X = Y$ via a one-to-one mapping.  Since $H \left( X, Y \right) = 0$ iff $X$ is a function of $Y$ and $H \left( Y, X \right) = 0$ iff $Y$ is a function of $X$, $\rho \left( X, Y \right)$ can only equal zero if and only if $X = Y$.  This satisfies the third and only remaining condition of a metric over all $x$ and $y$; therefore, for cases where $X$ and $Y$ are related by a one-to-one mapping, $\rho \left( X, Y \right)$ is a metric over all $x$ and $y$.





























\end{document}