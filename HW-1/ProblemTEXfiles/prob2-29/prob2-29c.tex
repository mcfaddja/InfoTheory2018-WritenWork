\documentclass[ClusteringConnectionsMAIN.tex]{subfiles}




\begin{document}
	

\prob{2.29 c}  Starting with $\Hent{X, Y, Z} - \Hent{X, Y}$, and noting that $\Hent{X, Y} = \Hent{X} + \Hent{Y \mid X}$ and $\Hent{X, Y, Z} = \Hent{X} + \Hent{Y \mid X} + \Hent{Z \mid X, Y}$, we have

\begin{align} \label{c2p29eq1}
\Hent{X, Y, Z} - \Hent{X, Y} &= \cancel{ \Hent{X} } + \cancel{ \Hent{Y \mid X} } + \Hent{Z \mid X, Y} - \left( \cancel{ \Hent{X} } + \cancel{ \Hent{Y \mid X} } \right)   \notag \\
&= \Hent{Z \mid X, Y}   \tag{2.29-1}
\end{align}

By the definition of conditional mutual information, we have

\begin{align}
&\I{Y; Z \mid X} = \Hent{Z, X} - \Hent{Z \mid X, Y}  \notag \\
&\Longrightarrow \Hent{Z \mid X, Y} = \Hent{Z, X} - \I{Y; Z \mid X}  \notag
\end{align}

Applying the above relation to the expression in \ref{c2p29eq1} gives

\begin{align} \label{c2p29eq2}
\Hent{X, Y, Z} - \Hent{X, Y} &= \Hent{Z \mid X, Y}   \notag \\
&= \Hent{Z, X} - \I{Y; Z \mid X}   \notag \\
&\geq \Hent{Z, X}  \tag{2.29-2} 
\end{align}

since mutual information is always greater than or equal to zero.



































\end{document}