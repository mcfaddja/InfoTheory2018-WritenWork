\documentclass[ClusteringConnectionsMAIN.tex]{subfiles}




\begin{document}
	
\prob{2.24 a}  Consider a choice of four unique objects and specify a specific, arbitrarily picked object to be "special".  We can model this situation using two random variables $X$ and $Y$.  We define $X$ to be the random variable describing whether the "special" object was picked, so $\mathcal{X} = \left\{ 0, 1 \right\}$ with $X = 0$ if the "special" object was picked and $X = 1$ otherwise.  The entropy of this random variable, $H \left( X \right)$, is the entropy we seek to find, namely $H \left( 1 / 4 \right)$.  Additionally, we define $Y$ to be the random variable describing which of the three "non-special" items was picked, thus $\mathcal{Y} = \left\{ 0, 1, 2 \right\}$.  We define $Y = 0$ if the first "non-special" object was picked, $Y = 1$ for the second, and $Y = 2$ for the third. \newline

We can also define a third random variable $Z$ which describes choosing one of four unique objects, thus $\mathcal{Z} = \left\{ 0, 1, 2, 3\right\}$ and $\Pr \left[ Z = z \right] = 1 / 4$ for all $z \in \mathcal{Z}$.  Furthermore, we can equate $Z = X \, Y$, thus we can say

\begin{align} 
	H \left( Z \right) &= H \left( X, Y \right) \notag \\
	&= H \left( X \right) + H \left( Y \mid X \right) \notag
\end{align}

by also using the chain rule $H \left( X, Y \right) = H \left( X \right) + H \left( Y \mid X \right)$.  Without loss of generality, we also say that the "special" object is represented by $Z = 0$.  That is to say that  $H \left( X \right) = H \left( 1 / 4 \right) = H \left( Z = 0 \right)$.  \newline


We continue by appling the relation $H \left( X \right) = H \left( 1 / 4 \right)$ to the previous result gives us

\begin{align}
H \left( Z \right) &= H \left( X \right) + H \left( Y \mid X \right) \notag \\
&= H \left( 1/4 \right) + H \left( Y \mid X \right) \notag
\end{align}

which becomes

\begin{align} \label{c2p24eq1}
\longrightarrow H \left( 1/4 \right) &= H \left( Z \right) - H \left( Y \mid X \right) \notag \\
&= H \left( Z \right) - \sum_{\substack{x, y \\ x = 1 \\ y \in \mathcal{Y}}} \biggl\{ p \left( x, y \right) \log_2 \bigl[ p \left( y | x \right) \bigr] \biggr\}  \notag \\
&= H \left( Z \right) - \sum_{\substack{x, y \\ x = 1 \\ y \in \mathcal{Y}}} \biggl\{ \sum_{z} \bigl\{ p \left( z \right) \, p \left( y \mid x \right) \bigr\} \log_2 \bigl[ p \left( y | x \right) \bigr] \biggr\}  \notag \\
&= H \left( Z \right) - \sum_{\substack{x, y \\ x = 1 \\ y \in \mathcal{Y}}} \biggl\{ \sum_{z} \bigl\{ p \left( z \right) \bigr\} \; p \left( y \mid x \right) \log_2 \bigl[ p \left( y | x \right) \bigr] \biggr\}  \notag \\
&= H \left( Z \right) - \sum_{\substack{x, y \\ x = 1 \\ y \in \mathcal{Y}}} \biggl\{ \sum_z \bigl\{ p \left( z \right) \bigr\} \; p \left( y \mid x \right) \log_2 \bigl[ p \left( y | x \right) \bigr] \biggr\}  \notag \\
&= H \left( Z \right) - \sum_{\substack{z \\ z \neq 0}}  \biggl\{ \sum_{\substack{x, y \\ x = 1 \\ y \in \mathcal{Y}}} \bigl\{ p \left( z \right) \; p \left( y \mid x \right) \log_2 \bigl[ p \left( y | x \right) \bigr] \bigr\} \biggr\}  \notag \\
&= H \left( Z \right) - \sum_{\substack{z \\ z \neq 0}} \Biggl\{ p \left( z \right) \biggl( \sum_{\substack{x, y \\ x = 1 \\ y \in \mathcal{Y}}} \bigl\{ p \left( y \mid x \right) \log_2 \bigl[ p \left( y | x \right) \bigr] \bigr\} \biggr) \Biggr\} \tag{2.24-1}
\end{align}

through some simple rearranging along with noting the definition of $H \left( Y \mid X \right)$ and the fact that $Z = X Y$ implies that $p \left( z \right) = p \left( x, y \right)$.  Now, since $\Pr \left[ Z = z \right] = 1 / 4$ for all $z \in \mathcal{Z}$ we can compute the value of $H \left( Z \right)$ to be

\begin{align} \label{c2p24eq2}
H \left( Z \right) = \sum_{i=1}^4 \biggl\{ \frac{1}{4} \log_2 \left[ \frac{1}{4} \right] \biggr\} = 2  \tag{2.24-2}
\end{align}

Moreover, we can also evaluate the summation over $z$ in the second term of \ref{c2p24eq1} to obtain

\begin{align}
\sum_{\substack{z \\ z \neq 0}} \Biggl\{ p \left( z \right) \biggl( \sum_{\substack{x, y \\ x = 1 \\ y \in \mathcal{Y}}} \bigl\{ p \left( y \mid x \right) \log_2 \bigl[ p \left( y | x \right) \bigr] \bigr\} \biggr) \Biggr\} &= \frac{3}{4} \biggl( \sum_{\substack{x, y \\ x = 1 \\ y \in \mathcal{Y}}} \bigl\{ p \left( y \mid x \right) \log_2 \bigl[ p \left( y | x \right) \bigr] \bigr\} \biggr) \notag \\
\end{align}

Additionally, we can say that, when $X = 1$, $p \left( y | x \right) = 1/3$ for all $y \in \mathcal{Y}$.  Therefore, we can evaluate the summation over $x$ and $y$ in the previous result

\begin{align}
\sum_{\substack{z \\ z \neq 0}} \Biggl\{ p \left( z \right) \biggl( \sum_{\substack{x, y \\ x = 1 \\ y \in \mathcal{Y}}} \bigl\{ p \left( y \mid x \right) \log_2 \bigl[ p \left( y | x \right) \bigr] \bigr\} \biggr) \Biggr\} &= \frac{3}{4} \sum_{\substack{x, y \\ x = 1 \\ y \in \mathcal{Y}}} \biggl\{ p \left( y \mid x \right) \log_2 \bigl[ p \left( y | x \right) \bigr] \biggr\} \notag
\end{align}








































\end{document}