\documentclass{article}[12pt]
\setlength{\textheight}{8.75in}
\setlength{\topmargin}{-0.5in}
\setlength{\oddsidemargin}{-.25in}
\setlength{\evensidemargin}{0in}
\setlength{\textwidth}{7in}
\usepackage{amsfonts, amsmath, amsthm, amssymb, mathrsfs, graphicx, fancyhdr, cancel, latexsym, multicol}

%\usepackage{savesym}
%\usepackage{amsmath}
%\savesymbol{iint}
%\usepackage{txfonts}
%\restoresymbol{TXF}{iint}

\pagestyle{fancy}
\lhead{Jonathan McFadden \\ Homework 1}
\rhead{Winter 2018 \\ TCSS 580}
\headsep = 22pt 
\headheight = 22.55pt



% BEGIN PRE-AMBLE


% Setup equation numbering 
\numberwithin{equation}{subsection} 

%Equation Numbering Shortcut Commands
\newcommand{\numbch}[1]{\setcounter{section}{#1} \setcounter{equation}{0}}
\newcommand{\numbpr}[1]{\setcounter{subsection}{#1} \setcounter{equation}{0}}
\newcommand{\note}{\textbf{NOTE:  }}

%Formatting shortcut commands
\newcommand{\chap}[1]{\begin{center}\begin{Large}\textbf{\underline{#1}}\end{Large}\end{center}}
\newcommand{\prob}[1]{\textbf{\underline{Problem #1):}}}
\newcommand{\sol}[1]{\textbf{\underline{Solution #1):}}}
\newcommand{\MMA}{\emph{Mathematica }}

%Text Shortcut Command
\newcommand{\s}[1]{\emph{Side #1}}

% Math shortcut commands
\newcommand{\pd}[2]{\frac{\partial #1}{\partial #2}}
\newcommand{\pdn}[3]{\frac{\partial^{#1} #2}{\partial #3^{#1}}}
\newcommand{\infint}{\int_{-\infty}^\infty}
\newcommand{\infiint}{\iint_{-\infty}^\infty}
\newcommand{\infiiint}{\iiint_{-\infty}^\infty}
\newcommand{\dint}[2]{\int_{#1}^{#2}}
\newcommand{\dd}[1]{\textrm{d#1}}
\newcommand{\ddd}[1]{\textrm{d}#1}
\renewcommand{\Re}{\mathbb{R}}

%Math Text
\newcommand{\csch}{\text{ csch}}

%Physics Shortcut Commands
\newcommand{\h}{\mathcal{H}}
\newcommand{\Z}{\mathcal{Z}}


% END PRE-AMBLE




\begin{document}


\begin{center}
	\begin{Huge}
		\textbf{ \underline{Homework \#1} }
	\end{Huge}
\end{center}

\vspace{0.1in}

\begin{center}
	\begin{Large}
		\textbf{\emph{Jonathan McFadden}}
	\end{Large}
\end{center}

\vspace{0.2in}

\begin{center}
	\begin{large}
		TCSS - 580  :  Winter 2018 \\
	\end{large}
	Information Theory
\end{center}

\vspace{1.25in}



\begin{flushleft}



\prob{2.3}  Let $\mathbb{P}^n$ be the set of all $n$-dimensional probability vectors, with elements $\vec{p} \in \mathbb{P}^n$ defined as $\vec{p} = \left( p_1, p_2, \dots, p_i, \dots, p_n \right)$ for $i \in \mathbb{Z}^+ \ni i \leq n$.  By the definition of a probability space, we must have

\begin{align} \label{c2p3eq1}
\vec{p} \cdot \vec{1} = \sum_{i = 1}^n \left\{ p_i \right\} = 1  \; \; , \; \; \, \forall \vec{p} \in \mathbb{P}^n, \tag{2.3-1}
\end{align}

where vector $\vec{1}$ is defined as $\vec{1} = \left( q_1, q_2, \dots, q_k, \dots, q_n \right) \in \mathbb{Z}^n$ with $q_k = 1, \forall k \in \left[ 1, n \right]$ (\emph{where the interval} $\left[ 1, n \right]$ \emph{is defined such that} $\left[ 1, n \right] \subseteq \mathbb{Z}^+$). Futhermore, the definition of a probability space also requires that, for any $\vec{p} \in \mathbb{P}^n$, the elements of $\vec{p}$ (\emph{the} $p_i \in \vec{p}$ \emph{such that} $i \in \mathbb{Z}^+ \ni i \leq n$) satisfy the condition

\begin{align} \label{c2p3eq2}
p_i \geq 0  \tag{2.3-2} 
\end{align}

for all $i \in \mathbb{Z}^+ \ni i \leq n$. \newline


The expression in \ref{c2p3eq1} guarantees that the $p_i$ of any $\vec{p} \in \mathbb{P}^n$ satisfy the bound $0 \leq p_i \leq 1$ where $i \in \mathbb{Z}^+ \ni i \leq n$.  Therefore, the relation

\begin{align} \label{c2p3eq3}
p_i \log_2 \left[ p_i \right] \geq 0  \tag{2.3-3}
\end{align}

holds for all $p_i$ of any $\vec{p} \in \mathbb{P}^n$.  Moreover, for the cases where $p_i = 0$ or $p_i = 1$, it is clear that the expression in \ref{c2p3eq3} reduces to equality.  Specifically, the realation $p_i \log_2 \left[ p_i \right]$ becomes

\begin{align*} \label{c2p3eq4}
p_i \log_2 \left[ p_i \right] = 0  \tag{2.3-4}
\end{align*}

for the case where $p_i = 0$ or $p_i = 1$.  Moreover, the relation in \ref{c2p3eq4} also represents the \textbf{smallest} possible value/result for the expression $p_i \log_2 \left[ p_i \right]$.  That is to say, that when $p_i = 0$ or $p_i = 1$, then $p_i \log_2 \left[ p_i \right]$ is at a minimum.  \newline


The result in \ref{c2p3eq1}, makes it is clear that \emph{only} \textbf{\underline{ONE}} $p_i$ in each $\vec{p} \in \mathbb{P}^n$ may have the value $p_i = 1$; therefore the probability vectors $\vec{p} \in \mathbb{P}^n$ which result in a minimum value for $p_i \log_2 \left[ p_i \right]$ all have exactly one non-zero element with the non-zero element having a value of one.  This implies that there are only $n$ such probability vectors, $\vec{p}^\star$ within any $\mathbb{P}^n$.  Furthermore, the value of $H(X) = \sum_{i=1}^n \left\{ p_i \log_2 \left[ p_i \right] \right\}$ for any such $\vec{p}^\star$ is also zero.



\vspace{0.5in}

\prob{2.4 a}  Recall the chain-rule for conditional entropies of $X$ given $Y$,

\begin{align}  \label{c2p4eq1}
H \left( X \mid Y \right) = H \left( X, Y \right) - H \left( Y \right)  \tag{2.4-1}
\end{align}

We apply the expression in \ref{c2p4eq1} to the case of $g \left( X \right)$ given $X$ to obtain

\begin{align} 
H \left( g \left( X \right) \mid X \right) = H \left( g \left( X \right), X \right) - H \left( X \right) 
\end{align}

by rearranging the expression in the previous result as follows,

\begin{align} \label{c2p4eq2} 
H \left( g \left( X \right), X \right) = H \left( X \right) + H \left( g \left( X \right) \mid X \right)  \tag{2.4-2}
\end{align}

we obtain the desired result.



\vspace{0.5in}

\prob{2.4 b}  For any given value of $X$, we automatically know $g \left( X \right)$.  Therefore, the expression for $H \left( g \left( X \right), X \right)$ in \ref{c2p4eq2} becomes

\begin{align*} 
H \left( g \left( X \right), X \right) = H \left( X \right)  
\end{align*}

which is the desired result.



\vspace{0.5in}

\prob{2.4 c}  Recalling the expression for the conditional entropy chain rule in \ref{c2p4eq1} and using it for the case where $X = X$ and $Y = g \left( X \right)$ yields the result

\begin{align*} 
H \left( X \mid g \left( X \right) \right) = H \left( X, g \left( X \right) \right) - H \left( g \left( X \right) \right)  
\end{align*}

rearranging the above expression yields 

\begin{align} \label{c2p4eq3}
H \left( X, g \left( X \right) \right) = H \left( g \left( X \right) \right) + H \left( X \mid g \left( X \right) \right)  \tag{2.4-3}
\end{align}

we obtain the desired result.



\vspace{0.5in}

\prob{2.4 d}  For any arbitrary function, $g \left( X \right)$, of a random variable $X$, the entropy $H \left( X \mid g \left( X \right)\right)$ satisfies the condition

\begin{align} \label{c2p4eq4}
H \left( X \mid g \left( X \right)\right) \geq 0  \tag{2.4-4}
\end{align}

for the case where $g \left( X \right)$ is one-to-one, the relation in \ref{c2p4eq4} simplifies to

\begin{align*}
H \left( X \mid g \left( X \right)\right) = 0
\end{align*}

Applying the relation in \ref{c2p4eq4} to the expression in \ref{c2p4eq3} yields

\begin{align*}
H \left( X, g \left( X \right) \right) &= H \left( g \left( X \right) \right) + H \left( X \mid g \left( X \right) \right) \\
&\geq H \left( g \left( X \right) \right) + H \left( X \mid g \left( X \right) \right) - H \left( X \mid g \left( X \right)\right) \\
&\geq H \left( g \left( X \right) \right)
\end{align*}











































































































\end{flushleft}
\end{document}