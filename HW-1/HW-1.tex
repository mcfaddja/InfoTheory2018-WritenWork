\documentclass{article}[12pt]
\setlength{\textheight}{8.75in}
\setlength{\topmargin}{-0.5in}
\setlength{\oddsidemargin}{-.25in}
\setlength{\evensidemargin}{0in}
\setlength{\textwidth}{7in}
\usepackage{amsfonts, amsmath, amsthm, amssymb, mathrsfs, graphicx, fancyhdr, cancel, latexsym, multicol}

%\usepackage{savesym}
%\usepackage{amsmath}
%\savesymbol{iint}
%\usepackage{txfonts}
%\restoresymbol{TXF}{iint}

\pagestyle{fancy}
\lhead{Jonathan McFadden \\ Homework 1}
\rhead{Winter 2018 \\ TCSS 580}
\headsep = 22pt 
\headheight = 22.55pt



% BEGIN PRE-AMBLE


% Setup equation numbering 
\numberwithin{equation}{subsection} 

%Equation Numbering Shortcut Commands
\newcommand{\numbch}[1]{\setcounter{section}{#1} \setcounter{equation}{0}}
\newcommand{\numbpr}[1]{\setcounter{subsection}{#1} \setcounter{equation}{0}}
\newcommand{\note}{\textbf{NOTE:  }}

%Formatting shortcut commands
\newcommand{\chap}[1]{\begin{center}\begin{Large}\textbf{\underline{#1}}\end{Large}\end{center}}
\newcommand{\prob}[1]{\textbf{\underline{Problem #1):}}}
\newcommand{\sol}[1]{\textbf{\underline{Solution #1):}}}
\newcommand{\MMA}{\emph{Mathematica }}

%Text Shortcut Command
\newcommand{\s}[1]{\emph{Side #1}}

% Math shortcut commands
\newcommand{\pd}[2]{\frac{\partial #1}{\partial #2}}
\newcommand{\pdn}[3]{\frac{\partial^{#1} #2}{\partial #3^{#1}}}
\newcommand{\infint}{\int_{-\infty}^\infty}
\newcommand{\infiint}{\iint_{-\infty}^\infty}
\newcommand{\infiiint}{\iiint_{-\infty}^\infty}
\newcommand{\dint}[2]{\int_{#1}^{#2}}
\newcommand{\dd}[1]{\textrm{d#1}}
\newcommand{\ddd}[1]{\textrm{d}#1}
\renewcommand{\Re}{\mathbb{R}}

%Math Text
\newcommand{\csch}{\text{ csch}}

%Physics Shortcut Commands
\newcommand{\h}{\mathcal{H}}
\newcommand{\Z}{\mathcal{Z}}


% END PRE-AMBLE




\begin{document}


\begin{center}
	\begin{Huge}
		\textbf{ \underline{Homework \#1} }
	\end{Huge}
\end{center}

\vspace{0.1in}

\begin{center}
	\begin{Large}
		\textbf{\emph{Jonathan McFadden}}
	\end{Large}
\end{center}

\vspace{0.2in}

\begin{center}
	\begin{large}
		TCSS - 580  :  Winter 2018 \\
	\end{large}
	Information Theory
\end{center}

\vspace{1.0in}



\begin{flushleft}



\prob{2.3}  Let $\mathbb{P}^n$ be the set of all $n$-dimensional probability vectors.  Futhermore, let $\vec{p}$ be any element of $\mathbb{P}^n$ ($\vec{p} \in \mathbb{P}^n$) and define $\vec{p}$ as $\vec{p} = \left( p_1, p_2, \dots, p_i, \dots, p_n \right)$, where $i \in \mathbb{Z}^+ \ni i \leq n$.  By the definition of a probability space, we must have

\begin{align*} \label{c2p3eq1}
\vec{p} \cdot \vec{1} = \sum_{i=1}^n \left\{ p_i \right\} = 1  \; , \; \, \forall \vec{p} \in \mathbb{P}^n, \tag{2.3-1}
\end{align*}

where $\vec{1} = \left( 1, 1, \dots, 1, \dots, 1 \right)$ is the $n$-dimensional vector having the value $1$ for each of its components.  Additionally, we may equivalently state, that for $i \in \mathbb{Z}^+ \ni i \leq n$, the vector $\vec{1}$ can be defined as $\vec{1} = \left( q_1, q_2, \dots, q_i, \dots, q_n \right)$ where $q_i = 1, \forall i \in [1, n] \subseteq \mathbb{Z}^+$. \newline

Now, $\forall i \in \mathbb{Z}^+ \ni i \leq n$, it is clear that the relation $p_i \log_2 \left[ p_i \right] \geq 0$, holds.  Moreover, it is also apparent that the relation $p_i \log_2 \left[ p_i \right] \geq 0$ simplifies to equality ($p_i \log_2 \left[ p_i \right] = 0$) for the cases where either $p_i = 0$ or $p_i = 1$.  Now, from the result in \ref{c2p3eq1}, it is clear that \emph{only} \textbf{\underline{ONE}} $p_i$ in each $\vec{p} \in \mathbb{P}^n$ may have the value $p_i = 1$.  Moreover, for the case where one element $p_i$ of $\vec{p}$ has the value $p_i = 1$, the expression in \ref{c2p3eq1} also requires that $p_j = 0, \forall j \in \left[ 1, n \right] \subseteq \mathbb{Z}^+ \ni j \neq i$.  That is to say, of one element $p_i$ of $\vec{p}$ is such that $p_i = 1$, then all other elements of $\vec{p}$, $p_j$ (where $j \in \mathbb{Z}^+ \ni j \leq n, j \neq i$), must be zero ($p_i = 0, \forall j \in \ left[ 1, n \right] \subseteq \mathbb{Z}^+ \ni j \neq i$).   \newline










































































































\end{flushleft}
\end{document}